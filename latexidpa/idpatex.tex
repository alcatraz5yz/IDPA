\documentclass{article}
\usepackage[utf8]{inputenc}
\usepackage[T1]{fontenc}
\usepackage[ngerman]{babel}
\usepackage[margin=1in,
includefoot]{geometry}%f?r rand des Textes/dokumentes
\usepackage{lipsum}%fillertext latin nonsense
\usepackage[margin=1in,left=1.5in,includefoot]{geometry}%f?r rand des Textes/dokumentes
\usepackage[hidelinks]{hyperref} % Allows for clickable references
% Tables preamble
\usepackage[none]{hyphenat} % Stops breaking up words in a table
\usepackage[english]{babel}
\usepackage{float} % Allows for control of float positions.
\usepackage{listings}
\usepackage{color}
%%%%%%%%%%%%%%%%%%%%%%%%%%%%%%%%%%%%%%%%%%%%%%%%%%%%%%%%%%%%%%%%%%%%%%%%%%%%%%%%
%f?r codeblocks
\lstdefinestyle{mystyle}{
    backgroundcolor=\color{backcolour},
    commentstyle=\color{codegreen},
    keywordstyle=\color{magenta},
    numberstyle=\tiny\color{codegray},
    stringstyle=\color{codepurple},
    basicstyle=\footnotesize,
    breakatwhitespace=false,
    breaklines=true,
    captionpos=b,
    keepspaces=true,
    numbers=left,
    numbersep=5pt,
    showspaces=false,
    showstringspaces=false,
    showtabs=false,
    tabsize=
}
\definecolor{codegreen}{rgb}{0,0.6,0}
\definecolor{codegray}{rgb}{0.5,0.5,0.5}
\definecolor{codepurple}{rgb}{0.58,0,0.82}
\definecolor{backcolour}{rgb}{0.95,0.95,0.92}
\lstset{style=mystyle}
%%%%%%%%%%%%%%%%%%%%%%%%%%%%%%%%%%%%%%%%%%%%%%%%%%%%%%%%%%%%%%%%%%%%%%%%%%%%%%%%
\usepackage{multicol}
%shortcut for fettew?rter machen geht auch mit bfseries, wenn man das hier hat,
%muss man nur \rowstyle $ ^ ^ ^ ^  diese package ist gar nicht n?tig bei 2 table
\usepackage{array}
\newcolumntype{$}{>{\global\let\currentrowstyle\relax}}
\newcolumntype{^}{>{\currentrowstyle}}
\newcommand{\rowstyle}[1]{\gdef\currentrowstyle{#1} #1\ignorespaces}
%%%%%%%%%%%%%%%%%%%%%%%%%%%%%%%%%%%%%%%%%%%%%%%%%%%%%%%%%%%%%%%%%%%%%%%%%%%%%%%%
% Graphics preamble Allows you to import images
\usepackage{graphicx}
%%%%%%%%%%%%%%%%%%%%%%%%%%%%%%%%%%%%%%%%%%%%%%%%%%%%%%%%%%%%%%%%%%%%%%%%%%%%%%%%
\usepackage{float} % Allows for control of float positions.
%%%%%%%%%%%%%%%%%%%%%%%%%%%%%%%%%%%%%%%%%%%%%%%%%%%%%%%%%%%%%%%%%%%%%%%%%%%%%%%%
% Math preamble Allows us to write chemistry equations!
\usepackage{mhchem}
%f?r mathe 1/2 z.b wird richtig gezeigt.==> \sfrac{1}{2} ohne \frac{1}{2}
\usepackage{xfrac}
%%%%%%%%%%%%%%%%%%%%%%%%%%%%%%%%%%%%%%%%%%%%%%%%%%%%%%%%%%%%%%%%%%%%%%%%%%%%%%%%
% Bibliography preamble f?r Referenzen hier mit bibtex reihenfolge hyperlinking zu refer.
\usepackage[numbers,sort&compress]{natbib}
%%%%%%%%%%%%%%%%%%%%%%%%%%%%%%%%%%%%%%%%%%%%%%%%%%%%%%%%%%%%%%%%%%%%%%%%%%%%%%%%
% Bullet preamble for \begin{itemize} \item a f?r subbullet nochmals beginn und item
\renewcommand{\labelitemi}{$\bullet$}
\renewcommand{\labelitemii}{$\diamond$}
\renewcommand{\labelitemiii}{$\circ$}
%%%%%%%%%%%%%%%%%%%%%%%%%%%%%%%%%%%%%%%%%%%%%%%%%%%%%%%%%%%%%%%%%%%%%%%%%%%%%%%%
% f?r kopf und fusszeile }%macht f?r Kopf/Fusszeile strich+ man kann da dann schreiben
\usepackage{fancyhdr}
\pagestyle{fancy}
\fancyhead{}
\fancyfoot{}
\fancyfoot[R]{ \thepage\ }
\renewcommand{\headrulewidth}{0pt}
\renewcommand{\footrulewidth}{0pt}
%%%%%%%%%%%%%%%%%%%%%%%%%%%%%%%%%%%%%%%%%%%%%%%%%%%%%%%%%%%%%%%%%%%%%%%%%%%%%%%%
\begin{document}
\begin{titlepage}
    \begin{center}
    \line(1,0){300} \\ %macht eine linie mit durchmesser 300
    [3mm]%h?he der Linie
    \huge{\bfseries creating and hosting a flask web application} \\
    [2mm]%h?he der Linie
    \line(1,0){200} \\ %macht eine linie mit durchmesser 200
    [1.5cm]%abstand h?he
    \textsc{\LARGE IDPA} \\
    [0.75cm]
    \textsc{\Large Please fork me on github on alcatraz5yzee} \\
    [9cm]
    \end{center}
    \begin{flushright}
    \textsc{\large Allan Kueng\\
    May 19, 2017 \\}
    \end{flushright}
\end{titlepage}


% Front matter stuff
\pagenumbering{roman}
\section*{Summary}
\addcontentsline{toc}{section}{\numberline{}Summary}%f?gt summary mit i zu table of contents
=========text
\cleardoublepage



\section*{Acknowledgements}
\addcontentsline{toc}{section}{\numberline{}Acknowledgements}
Thanks to Latex creators.



\begin{lstlisting}
import numpy as np

def incmatrix(genl1,genl2):
    m = len(genl1)
    n = len(genl2)
    M = None #to become the incidence matrix
    VT = np.zeros((n*m,1), int)  #dummy variable

    #compute the bitwise xor matrix
    M1 = bitxormatrix(genl1)
    M2 = np.triu(bitxormatrix(genl2),1)

    for i in range(m-1):
        for j in range(i+1, m):
            [r,c] = np.where(M2 == M1[i,j])
            for k in range(len(r)):
                VT[(i)*n + r[k]] = 1;
                VT[(i)*n + c[k]] = 1;
                VT[(j)*n + r[k]] = 1;
                VT[(j)*n + c[k]] = 1;

                if M is None:
                    M = np.copy(VT)
                else:
                    M = np.concatenate((M, VT), 1)

                VT = np.zeros((n*m,1), int)

    return M
\end{lstlisting}


\begin{lstlisting}
	def hello world
\end{lstlisting}

\cleardoublepage



% This is table of contents stuff
\renewcommand{\contentsname}{Inhaltsverzeichnis}
\tableofcontents
\thispagestyle{empty}%keine Seitenzahl mehr
\cleardoublepage




% List of pictures
\listoffigures
\addcontentsline{toc}{section}{\numberline{}List of Figures}
\cleardoublepage


%list of tables
\listoftables
\addcontentsline{toc}{section}{\numberline{}List of Tables}
\cleardoublepage




% This is main body stuff nach einf?hrung, hauptteil
\pagenumbering{arabic}
\setcounter{page}{1}%nach thispagestyle wieder mit seitenzahl 1 anfangen




\section{Introduction}\label{sec:intro}%label for hyperlink to reference
\section{html}

\cleardoublepage


\section{css}
\cleardoublepage


\section{javascript}
\cleardoublepage



\section{text editor}
\cleardoublepage



\section{heroku}
\renewcommand{\footrulewidth}{1pt}
\fancyfoot[c]{hello}

\cleardoublepage



\section{signup/login}
\cleardoublepage



\section{deployment}
\cleardoublepage



\section{}
\cleardoublepage



\section{index}
\cleardoublepage



\section{stream}
\cleardoublepage

\section{letsplay}
\cleardoublepage

\section{about}
\cleardoublepage

\section{logout}
\cleardoublepage

\section{}
\cleardoublepage

\section{}
\cleardoublepage



\section{References}





\cleardoublepage




% Appendix starts here
\appendix
\section{Backmatter Words}
Here are the specific links for all the important websites and my code



\end{document}
