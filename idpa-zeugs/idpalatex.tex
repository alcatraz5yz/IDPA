\documentclass{article}
\usepackage[utf8]{inputenc}
\usepackage[T1]{fontenc}
\usepackage[ngerman]{babel}
\usepackage[margin=1in,
includefoot]{geometry}%f?r rand des Textes/dokumentes
\usepackage{lipsum}%fillertext latin nonsense
\usepackage[margin=1in,left=1.5in,includefoot]{geometry}%f?r rand des Textes/dokumentes
\usepackage[hidelinks]{hyperref} % Allows for clickable references
% Tables preamble
\usepackage[none]{hyphenat} % Stops breaking up words in a table
\usepackage[english]{babel}
\usepackage{float} % Allows for control of float positions.
\usepackage{listings}
\usepackage{color}

%%%%%%%%%%%%%%%%%%%%%%%%%%%%%%%%%%%%%%%%%%%%%%%%%%%%%%%%%%%%%%%%%%%%%%%%%%%%%%%%
%f?r codeblocks
\lstdefinestyle{mystyle}{
    backgroundcolor=\color{backcolour},
    commentstyle=\color{codegreen},
    keywordstyle=\color{magenta},
    numberstyle=\tiny\color{codegray},
    stringstyle=\color{codepurple},
    basicstyle=\footnotesize,
    breakatwhitespace=false,
    breaklines=true,
    captionpos=b,
    keepspaces=true,
    numbers=left,
    numbersep=5pt,
    showspaces=false,
    showstringspaces=false,
    showtabs=false,
    tabsize=
}
\definecolor{codegreen}{rgb}{0,0.6,0}
\definecolor{codegray}{rgb}{0.5,0.5,0.5}
\definecolor{codepurple}{rgb}{0.58,0,0.82}
\definecolor{backcolour}{rgb}{0.95,0.95,0.92}
\lstset{style=mystyle}
%%%%%%%%%%%%%%%%%%%%%%%%%%%%%%%%%%%%%%%%%%%%%%%%%%%%%%%%%%%%%%%%%%%%%%%%%%%%%%%%
\usepackage{multicol}
%shortcut for fettew?rter machen geht auch mit bfseries, wenn man das hier hat,
%muss man nur \rowstyle $ ^ ^ ^ ^  diese package ist gar nicht n?tig bei 2 table
\usepackage{array}
\newcolumntype{$}{>{\global\let\currentrowstyle\relax}}
\newcolumntype{^}{>{\currentrowstyle}}
\newcommand{\rowstyle}[1]{\gdef\currentrowstyle{#1} #1\ignorespaces}
%%%%%%%%%%%%%%%%%%%%%%%%%%%%%%%%%%%%%%%%%%%%%%%%%%%%%%%%%%%%%%%%%%%%%%%%%%%%%%%%
% Graphics preamble Allows you to import images
\usepackage{graphicx}
%%%%%%%%%%%%%%%%%%%%%%%%%%%%%%%%%%%%%%%%%%%%%%%%%%%%%%%%%%%%%%%%%%%%%%%%%%%%%%%%
\usepackage{float} % Allows for control of float positions.
%%%%%%%%%%%%%%%%%%%%%%%%%%%%%%%%%%%%%%%%%%%%%%%%%%%%%%%%%%%%%%%%%%%%%%%%%%%%%%%%
% Math preamble Allows us to write chemistry equations!
\usepackage{mhchem}
%f?r mathe 1/2 z.b wird richtig gezeigt.==> \sfrac{1}{2} ohne \frac{1}{2}
\usepackage{xfrac}
%%%%%%%%%%%%%%%%%%%%%%%%%%%%%%%%%%%%%%%%%%%%%%%%%%%%%%%%%%%%%%%%%%%%%%%%%%%%%%%%
% Bibliography preamble f?r Referenzen hier mit bibtex reihenfolge hyperlinking zu refer.
\usepackage[numbers,sort&compress]{natbib}
%%%%%%%%%%%%%%%%%%%%%%%%%%%%%%%%%%%%%%%%%%%%%%%%%%%%%%%%%%%%%%%%%%%%%%%%%%%%%%%%
% Bullet preamble for \begin{itemize} \item a f?r subbullet nochmals beginn und item
\renewcommand{\labelitemi}{$\bullet$}
\renewcommand{\labelitemii}{$\diamond$}
\renewcommand{\labelitemiii}{$\circ$}
%%%%%%%%%%%%%%%%%%%%%%%%%%%%%%%%%%%%%%%%%%%%%%%%%%%%%%%%%%%%%%%%%%%%%%%%%%%%%%%%
% f?r kopf und fusszeile }%macht f?r Kopf/Fusszeile strich+ man kann da dann schreiben
\usepackage{fancyhdr}
\pagestyle{fancy}
\fancyhead{}
\fancyfoot{}
\fancyfoot[R]{ \thepage\ }
\renewcommand{\headrulewidth}{0pt}
\renewcommand{\footrulewidth}{0pt}
%%%%%%%%%%%%%%%%%%%%%%%%%%%%%%%%%%%%%%%%%%%%%%%%%%%%%%%%%%%%%%%%%%%%%%%%%%%%%%%%
\begin{document}
\section{\"Ubersicht}
\subsection{Die Programmiersprachen}
eine website kann man eigentlich mit nur CSS und HTML erstellen, da ich es aber etwas komplexer haben will, muss ich auch noch die sprachen Python und Javascript hinzuf\"ugen.
\subsection{jquery}
Mit jquery mache ich die spiele und alle animationen. Jquery ist ein Framework welches auf Javascript beruht
\subsection{Datenbank}
Da ich noch anf\"anger bin, habe ich mich f\"ur die Lite version der Databases entschieden n\"amlich sqlite
\subsection{Hochladen}
Zum Gl\"uck gibt es einen Ort wo ich meine seite gratis Hochladen kann, dieser Ort heisst Heroku.
Heroku ist ein Service der es mir erm\"oglicht komplexere seiten hochzuladen und diese auch zu warten.
\subsection{Latex}
Ich schreibe mein Project und diese Zwischenarbeit nicht wie alle anderen mit Word, sondern mit der Programmiersprache Latex.
\subsection{github}
Ich Lade meinen ganzen Code nat\"urlich auf Github um ihn mit anderen Teilen zu k\"onnen, und ihn abzusichern.
Da Github mit version control arbeitet ist es einfach meinen Code herunterzuladen und ihn zu ver\"andern.

\subsection{schon gemacht}
Sprachen lernen (html/ html mit flask/ css/Python flask/)
\\*
Grundstruktur der Seiten mit css/html
\\*
Git/Github Basics gelernt und code auf github gespeichert
\\*
Heroku Hochladen mit Procfile/requirements.txt
\\*
Seite zum Zeichnen mit jquery
\\*
Seite zum Posten mit flask







\subsection{noch zu machen}
Post teil noch verbessern
\\*
Hintergrund noch verbessern
\\*
evtl noch bootstrap einf\"ugen
\\*
game noch einf\"ugen
\\*
login evtl 3d
\\*
evtl noch mehr sicherheit beim login
\\*
evtl testfiles
\\*
Evtl mit mail kontaktieren wenn signup
\\*
Helpertools verwenden z.B. bower/gulp/sass







\renewcommand{\footrulewidth}{1pt}
\fancyfoot[c]{Zusammenfassung}
\cleardoublepage

\end{document}
